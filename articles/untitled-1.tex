\documentclass[mathserif,11pt]{beamer}


\setbeamercolor{block title}{bg=red!30,fg=black}

\mode<presentation>
{
\usetheme{default}
}


\title[] % (optional, use only with long paper titles)
{ProgPilot : Static Analyzer}
% \subtitle
% {Include Only If Paper Has a Subtitle}

\author[Therond, Therond]
{
  Eric~Therond
}

% Includes for frame package
\usepackage{framed,color}
\definecolor{shadecolor}{rgb}{0.5,0.5,0.5}

% Make a custom block
\newenvironment<>{customBlock}[1]{%
  \begin{actionenv}#2%
      \def\insertblocktitle{#1}%
      \par%
      \mode<presentation>{%
        \setbeamercolor{block title}{fg=white,bg=orange!20!black}
       \setbeamercolor{block body}{fg=black,bg=olive!50}
       \setbeamercolor{itemize item}{fg=orange!20!black}
       \setbeamertemplate{itemize item}[triangle]
     }%
      \usebeamertemplate{block begin}}
    {\par\usebeamertemplate{block end}\end{actionenv}}


% Define new environments for mdframed package
\usepackage{tikz}
\usetikzlibrary{shapes.callouts,shadows, calc}
\usepackage{listings}
\usepackage{lmodern}
\usepackage[framemethod=tikz]{mdframed}
\usepackage{amsmath}
\usepackage{amssymb}
\usepackage{algpseudocode} 

\tikzset{note/.style={rectangle callout, rounded corners,fill=gray!20,drop shadow,font=\footnotesize}}    

\newcommand{\tikzmark}[1]{\tikz[overlay,remember picture] \node (#1) {};}    

\newmdenv[tikzsetting={draw=black,fill=white,fill opacity=0.7, line width=4pt},backgroundcolor=none,leftmargin=0,rightmargin=0,innertopmargin=4pt,skipbelow=\baselineskip,%
skipabove=\baselineskip]{TitleBox}
%http://tex.stackexchange.com/questions/38281/transparent-background-for-mdframed-environment

\usetikzlibrary{shapes.geometric, arrows, positioning}

\tikzstyle{startstop} = [rectangle, rounded corners, minimum width=3cm, minimum height=1cm,text centered, draw=black, fill=red!30]
\tikzstyle{io} = [trapezium, trapezium left angle=70, trapezium right angle=110, minimum width=3cm, minimum height=1cm, text centered, draw=black, fill=blue!30]
\tikzstyle{process} = [rectangle, minimum width=3cm, minimum height=1cm, text centered, text width=3cm, draw=black, fill=orange!30]
\tikzstyle{decision} = [diamond, minimum width=3cm, minimum height=1cm, text centered, draw=black, fill=green!30]
\tikzstyle{arrow} = [thick,->,>=stealth]


\usetikzlibrary{shapes.geometric, arrows}
\tikzstyle{node} = [rectangle, rounded corners, minimum width=2cm, minimum height=0.5cm,text centered, draw=black]
\tikzstyle{arrow} = [thick,->,>=stealth]

\newcounter{image}
\setcounter{image}{1}

\makeatletter
\newenvironment{btHighlight}[1][]
{\begingroup\tikzset{bt@Highlight@par/.style={#1}}\begin{lrbox}{\@tempboxa}}
{\end{lrbox}\bt@HL@box[bt@Highlight@par]{\@tempboxa}\endgroup}

\newcommand\btHL[1][]{%
  \begin{btHighlight}[#1]\bgroup\aftergroup\bt@HL@endenv%
}
\def\bt@HL@endenv{%
  \end{btHighlight}%   
  \egroup
}
\newcommand{\bt@HL@box}[2][]{%
  \tikz[#1]{%
    \pgfpathrectangle{\pgfpoint{0pt}{0pt}}{\pgfpoint{\wd #2}{\ht #2}}%
    \pgfusepath{use as bounding box}%
    \node[anchor=base west,rounded corners, fill=green!30,outer sep=0pt,inner xsep=0.2em, inner ysep=0.1em,  #1](a\theimage){\usebox{#2}};
  }%
   %\tikzmark{a\theimage} <= can be used, but it leads to a spacing problem
   % the best approach is to name the previous node with (a\theimage)
 \stepcounter{image}
}
\makeatother


\lstset{language=C++,
                basicstyle=\footnotesize\ttfamily,
                keywordstyle=\tiny\color{blue}\ttfamily,
                moredelim=**[is][\btHL]{`}{`},
}

%----------------------------------------------------------------------------------------
\begin{document}

\usebackgroundtemplate{\includegraphics[width=1.0\paperwidth]{bg.jpg}}
  \begin{frame}[fragile] 

  \begin{TitleBox}
\begin{center}
    {\huge \inserttitle}\\
    Eric Therond\\
	contact@designsecurity.org\\
    http://www.designsecurity.org
\end{center}

   \end{TitleBox}
 \end{frame}

  \begin{frame}[fragile] 

  \begin{TitleBox}
\begin{center}
    {\huge Introduction}\\
\end{center}

   \end{TitleBox}
 \end{frame}

  \usebackgroundtemplate{\includegraphics[width=1.0\paperwidth]{bg.jpg}}
\begin{frame}[fragile] 
\setbeamercolor{block title}{use=structure,fg=white,bg=black!60}
\setbeamercolor{block body}{use=structure,fg=black,bg=white!60!white}
\begin{block}{What are the goals of static analysis ?}
\begin{itemize}
\item Find defaults in software
\item Improve dynamical analysis
\item Compute cyclomatic number
\item And priorize performance tests on code that have high cyclomatic number
\end{itemize}

\end{block}
\end{frame}

  \usebackgroundtemplate{\includegraphics[width=1.0\paperwidth]{bg.jpg}}
\begin{frame}[fragile] 
\setbeamercolor{block title}{use=structure,fg=white,bg=black!60}
\setbeamercolor{block body}{use=structure,fg=black,bg=white!60!white}
\begin{block}{Find lexical defaults}
\begin{lstlisting}[
    basicstyle=\tiny, %or \small or \footnotesize etc.
]
<?php
if(true)
return;
ellse
echo "error";
?>
\end{lstlisting}
\end{block}

\begin{block}{Find syntaxical defaults}
\begin{lstlisting}[
    basicstyle=\tiny, %or \small or \footnotesize etc.
]
<?php
for(if(true)$i =0;else $i=1; $i < 10; $i++)
{
echo "voila"
}
?>
\end{lstlisting}
\end{block}

\begin{block}{Find semantic defaults}
\begin{lstlisting}[
    basicstyle=\tiny, %or \small or \footnotesize etc.
]
<?php
class A
{
public $test1;
private $test2;
};

$var1 = new A;
$var1->test2 = "welcome";
?>
\end{lstlisting}
\end{block}
\end{frame}


  \usebackgroundtemplate{\includegraphics[width=1.0\paperwidth]{bg.jpg}}
\begin{frame}[fragile] 
\setbeamercolor{block title}{use=structure,fg=white,bg=black!60}
\setbeamercolor{block body}{use=structure,fg=black,bg=white!60!white}

\begin{tikzpicture}[node distance=2cm]
\node (start) [startstop] {Code Source};
\node (in1) [io, below of=start] {Abstract Syntax Tree};
\node (pro1) [process, below of=in1] {Control Flow Graph};
\node (dec1) [decision, below of=pro1, yshift=-0.5cm] {Decision 1};
\node (pro2a) [process, below of=dec1, yshift=-0.5cm] {Process 2a text text text text text text text text text text};
\node (pro2b) [process, right of=dec1, xshift=2cm] {Process 2b};
\node (out1) [io, below of=pro2a] {Output};
\node (stop) [startstop, below of=out1] {Stop};

\draw [arrow] (start) -- (in1);
\draw [arrow] (in1) -- (pro1);
\draw [arrow] (pro1) -- (dec1);
\draw [arrow] (dec1) -- node[anchor=east] {yes} (pro2a);
\draw [arrow] (dec1) -- node[anchor=south] {no} (pro2b);
\draw [arrow] (pro2b) |- (pro1);
\draw [arrow] (pro2a) -- (out1);
\draw [arrow] (out1) -- (stop);
\end{tikzpicture}
\end{frame}


  \usebackgroundtemplate{\includegraphics[width=1.0\paperwidth]{bg.jpg}}
\begin{frame}[fragile] 
\setbeamercolor{block title}{use=structure,fg=white,bg=black!60}
\setbeamercolor{block body}{use=structure,fg=black,bg=white!60!white}

\begin{tikzpicture}
\coordinate (a) at (0,0);
\coordinate (b) at (6,0);
\draw[->, >=latex, blue!20!white, line width=15pt]   (a) to node[black]{text} (b) ;
\end{tikzpicture}
\end{frame}

\usebackgroundtemplate{\includegraphics[width=1.0\paperwidth]{bg.jpg}}
  \begin{frame}[fragile] 

  \begin{TitleBox}
\begin{center}
    {\huge Representations}\\
\end{center}

   \end{TitleBox}
 \end{frame}  

  \usebackgroundtemplate{\includegraphics[width=1.0\paperwidth]{bg.jpg}}
\begin{frame}[fragile] 
\setbeamercolor{block title}{use=structure,fg=white,bg=black!60}
\setbeamercolor{block body}{use=structure,fg=black,bg=white!60!white}
\begin{block}{Abstract syntax tree}
\begin{itemize}
\item A set of informations about syntax is stored in a tree.
\item Tree, graphs are very simple to use (traversors).
\item Preserve semantic of a program (unambiguous syntax).
\end{itemize}

\end{block}
\end{frame}


  \usebackgroundtemplate{\includegraphics[width=1.0\paperwidth]{bg.jpg}}
\begin{frame}[fragile] 
\setbeamercolor{block title}{use=structure,fg=white,bg=black!60}
\setbeamercolor{block body}{use=structure,fg=black,bg=white!60!white}
\begin{block}{Control flow graph}
\begin{itemize}
\item CFG is a graph where nodes represent basics blocks and edges represent jumps in the control flow graph.
\item Essential tool because all paths of a program during his execution could be represented.
\item Simple analysis of CFG could reveal defaults in a program (potentials optimizations like unreachable codes).
\end{itemize}

\end{block}
\end{frame}

% https://hgi.rub.de/media/emma/veroeffentlichungen/2013/09/18/backdoorDetection-CCS13.pdf
  \usebackgroundtemplate{\includegraphics[width=1.0\paperwidth]{bg.jpg}}
\begin{frame}[fragile] 
\setbeamercolor{block title}{use=structure,fg=white,bg=black!60}
\setbeamercolor{block body}{use=structure,fg=black,bg=white!60!white}
\begin{block}{Call graphs}
\begin{itemize}
\item Call graphs is a CFG where relashionships between functions are represented.
\item Nice for debugging performance issues.
\item But also to identify malicious code (like backdoors).
\end{itemize}

\end{block}
\end{frame}

\usebackgroundtemplate{\includegraphics[width=1.0\paperwidth]{bg.jpg}}
  \begin{frame}[fragile] 

  \begin{TitleBox}
\begin{center}
    {\huge Data flow analysis}\\
\end{center}

   \end{TitleBox}
 \end{frame}  

  \usebackgroundtemplate{\includegraphics[width=1.0\paperwidth]{bg.jpg}}
\begin{frame}[fragile] 
\setbeamercolor{block title}{use=structure,fg=white,bg=black!60}
\setbeamercolor{block body}{use=structure,fg=black,bg=white!60!white}
\begin{block}{What are the objectives of static analysis ?}
\begin{itemize}
\item Reaching definitions
\end{itemize}

\end{block}
\end{frame}

\usebackgroundtemplate{\includegraphics[width=1.0\paperwidth]{bg.jpg}}
\begin{frame}[fragile] 
\setbeamercolor{block title}{use=structure,fg=white,bg=black!60}
\setbeamercolor{block body}{use=structure,fg=black,bg=white!60!white}
\begin{block}{Reaching definitions equations}

$ 
\begin{array}{ll}
kill_{RD}([x := a]^l)  & = {(x, ?)} \\
 & \bigcup \{(x, l^{'}) | B^{{l}^{'}} $is an assignment to $x$ in $S_{*}\}\\
\end{array}
$\\

$ 
\begin{array}{ll}
gen_{RD}([x := a]^l) & = \{(x, l)\} 
\end{array}
$\\
\vspace{1cm}
$ RD_{entry}(l) \left \{
   \begin{array}{lr}
      \{ (x,?) | x \in FV(S_{*}) \} & $if l$ = init(S_{*}) \\
      \bigcup \{ RD_{exit}(l^{'}) | (l^{'}, l) \in flow(S_{*}) \} & $otherwise$
   \end{array}
   \right .$\\
$ RD_{exit}(l) \left \{
   \begin{array}{l}
      (RD_{entry}(l)$ \textbackslash $kill_{RD}(B^{l})) \bigcup gen_{RD}(B^{l}) \\
      $where $B^l \in blocks(S_{*})
   \end{array}
   \right .$\\
\end{block}
\end{frame}

% efficiency of data flow algorithms :
% http://www.cs.colostate.edu/~mstrout/CS553Fall06/slides/lecture12-fastdataflow.pdf
% http://www.cse.scu.edu/~atkinson/papers/icsm-01.pdf

% algorithms :
% http://www.ics.uci.edu/~lopes/teaching/inf212W12/readings/rep-analysis-soft.pdf
  \usebackgroundtemplate{\includegraphics[width=1.0\paperwidth]{bg.jpg}}
\begin{frame}[fragile] 
\setbeamercolor{block title}{use=structure,fg=white,bg=black!60}
\setbeamercolor{block body}{use=structure,fg=black,bg=white!60!white}
\begin{block}{Reaching definitions algorithm}
\begin{algorithmic}
\For{each node n}\\
$out[n] = gen[n]$
\EndFor\\
$change = true$
\While{change} \\
$change = false$
\For{each node n} \\
$in[n] = \bigcup out[p]$ where p is an immediate predecessor of n\\
$oldout = out[n]$\\
$out[n] = gen[n]\bigcup(in[n]-kill[n])$
\If{$out[n]\neq oldout$} \\
$change = true$
\EndIf
\EndFor
\EndWhile
\end{algorithmic}

\end{block}
\end{frame}

  \usebackgroundtemplate{\includegraphics[width=1.0\paperwidth]{bg.jpg}}
\begin{frame}[fragile] 
\setbeamercolor{block title}{use=structure,fg=white,bg=black!60}
\setbeamercolor{block body}{use=structure,fg=black,bg=white!60!white}
\begin{block}{Reaching definitions analysis}

\begin{center}
\begin{tikzpicture}[node distance=2cm]

\node (fact3) [node] {
\begin{lstlisting}[
    basicstyle=\tiny, %or \small or \footnotesize etc.
]
<?php
$a = "XSS!";
$b = 0;
\end{lstlisting}
};
\node (fact2) [node, below of=fact3] {
\begin{lstlisting}[
    basicstyle=\tiny, %or \small or \footnotesize etc.
]
if(rand()%2)
{
  $b = 0;
  $c = 0;
}
\end{lstlisting}
};
\node (fact1) [node, below of=fact2] {
\begin{lstlisting}[
    basicstyle=\tiny, %or \small or \footnotesize etc.
]
else
{
  $c = $a;
}
\end{lstlisting}
};
\node (fact0) [node, below of=fact1] {
\begin{lstlisting}[
    basicstyle=\tiny, %or \small or \footnotesize etc.
]
echo $c;
?>
\end{lstlisting}
};



\node (nnn0) [node, right of=fact0,xshift=2cm] {
\begin{lstlisting}[
    basicstyle=\tiny, %or \small or \footnotesize etc.
]
gen[0]{};
kill[0]{};
in[0]{};
out[0]{};
\end{lstlisting}
};



\node (nnn1) [node, right of=fact1,xshift=2cm] {
\begin{lstlisting}[
    basicstyle=\tiny, %or \small or \footnotesize etc.
]
gen[0]{};
kill[0]{};
in[0]{};
out[0]{};
\end{lstlisting}
};


\node (nnn2) [node, right of=fact2,xshift=2cm] {
\begin{lstlisting}[
    basicstyle=\tiny, %or \small or \footnotesize etc.
]
gen[0]{};
kill[0]{};
in[0]{};
out[0]{};
\end{lstlisting}
};


\node (nnn3) [node, right of=fact3,xshift=2cm] {
\begin{lstlisting}[
    basicstyle=\tiny, %or \small or \footnotesize etc.
]
gen[0]{};
kill[0]{};
in[0]{};
out[0]{};
\end{lstlisting}
};
\path[->]                 (fact3)    edge[bend left=10]                node[swap]  {}       (fact2);
\path[->]                 (fact2)    edge[bend left=10]                node[swap]  {}       (fact1);
\path[->]                 (fact1)    edge[bend left=10]                node[swap]  {}       (fact0);

\path[->, dashed]                 (fact3)    edge[bend left=4]                node[swap]  {}       (nnn3);
\path[->, dashed]                 (fact2)    edge[bend left=4]                node[swap]  {}       (nnn2);
\path[->, dashed]                 (fact1)    edge[bend left=4]                node[swap]  {}       (nnn1);
\path[->, dashed]                 (fact0)    edge[bend left=4]                node[swap]  {}       (nnn0);
\end{tikzpicture}
\end{center}

\end{block}
\end{frame}

% http://www.seas.harvard.edu/courses/cs252/2011sp/slides/Lec02-Dataflow.pdf
% slide 21
  \usebackgroundtemplate{\includegraphics[width=1.0\paperwidth]{bg.jpg}}
\begin{frame}[fragile] 
\setbeamercolor{block title}{use=structure,fg=white,bg=black!60}
\setbeamercolor{block body}{use=structure,fg=black,bg=white!60!white}
\begin{block}{Control flow graph}
\begin{center}
\begin{tabular}{|l|c|c|}
  \hline
 & \textbf{May} & \textbf{Must} \\
  \hline
\textbf{Forward} & Reaching definitions & Available expressions \\
  \hline
\textbf{Backward} & Live variables & Very busy expressions \\
  \hline
\end{tabular}
\end{center}

\end{block}
\end{frame}

\usebackgroundtemplate{\includegraphics[width=1.0\paperwidth]{bg.jpg}}
  \begin{frame}[fragile] 

  \begin{TitleBox}
\begin{center}
    {\huge Visualizations}\\
\end{center}

   \end{TitleBox}
 \end{frame}  

  \usebackgroundtemplate{\includegraphics[width=1.0\paperwidth]{bg.jpg}}
\begin{frame}[fragile] 
\setbeamercolor{block title}{use=structure,fg=white,bg=black!60}
\setbeamercolor{block body}{use=structure,fg=black,bg=white!60!white}
\begin{block}{What kinds of security tests can be automated ?}
\begin{lstlisting}
#include <future>
std::map<std::string,std::string> french
{{"hello","bonjour"},{"world","tout le monde"}};
int main()
{
std::string greet=french["hello"];
auto f=std::async(`[&]{std::cout << greet <<", ";}`);
`std::string audience=french["word"];`
f.get();
std::cout<<audience<<std::endl;
}
\end{lstlisting}
% To insert the annotation
\begin{tikzpicture}[remember picture,overlay]
% first annotation
\coordinate (aa) at ($(a1)+(5,5)$); % <= adjust this parameter to move the position of the annotation 
\node[note,draw,callout relative pointer={($(aa)-(9,3.35)$)},right] at (aa) {time consuming I/O};

%second annotation
\coordinate (bb) at ($(a2)+(3.25,3.25)$); % <= adjust this parameter to move the position of the annotation 
\node[note,draw,callout relative pointer={($(bb)-(7,1.1)$)},right] at (bb) {next lookup};
\end{tikzpicture}
\end{block}
\end{frame}


\end{document} 